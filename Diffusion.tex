% Für Bindekorrektur als optionales Argument "BCORfaktormitmaßeinheit", dann
% sieht auch Option "twoside" vernünftig aus
% Näheres zu "scrartcl" bzw. "scrreprt" und "scrbook" siehe KOMA-Skript Doku
\documentclass[12pt,a4paper,titlepage,headinclude,bibtotoc]{scrartcl}


%---- Allgemeine Layout Einstellungen ------------------------------------------

% Für Kopf und Fußzeilen, siehe auch KOMA-Skript Doku
\usepackage[komastyle]{scrpage2}
\pagestyle{scrheadings}
\setheadsepline{0.5pt}[\color{black}]
\automark[section]{chapter}


%Einstellungen für Figuren- und Tabellenbeschriftungen
\setkomafont{captionlabel}{\sffamily\bfseries}
\setcapindent{0em}


%---- Weitere Pakete -----------------------------------------------------------
% Die Pakete sind alle in der TeX Live Distribution enthalten. Wichtige Adressen
% www.ctan.org, www.dante.de

% Sprachunterstützung
\usepackage[ngerman]{babel}

% Benutzung von Umlauten direkt im Text
% entweder "latin1" oder "utf8"
\usepackage[utf8]{inputenc}

% Pakete mit Mathesymbolen und zur Beseitigung von Schwächen der Mathe-Umgebung
\usepackage{latexsym,exscale,stmaryrd,amssymb,amsmath}

% Weitere Symbole
\usepackage[nointegrals]{wasysym}
\usepackage{eurosym}

% Anderes Literaturverzeichnisformat
%\usepackage[square,sort&compress]{natbib}

% Für Farbe
\usepackage{color}

% Zur Graphikausgabe
%Beipiel: \includegraphics[width=\textwidth]{grafik.png}
\usepackage{graphicx}

% Text umfließt Graphiken und Tabellen
% Beispiel:
% \begin{wrapfigure}[Zeilenanzahl]{"l" oder "r"}{breite}
%   \centering
%   \includegraphics[width=...]{grafik}
%   \caption{Beschriftung} 
%   \label{fig:grafik}
% \end{wrapfigure}
\usepackage{wrapfig}

% Mehrere Abbildungen nebeneinander
% Beispiel:
% \begin{figure}[htb]
%   \centering
%   \subfigure[Beschriftung 1\label{fig:label1}]
%   {\includegraphics[width=0.49\textwidth]{grafik1}}
%   \hfill
%   \subfigure[Beschriftung 2\label{fig:label2}]
%   {\includegraphics[width=0.49\textwidth]{grafik2}}
%   \caption{Beschriftung allgemein}
%   \label{fig:label-gesamt}
% \end{figure}
\usepackage{subfigure}

% Caption neben Abbildung
% Beispiel:
% \sidecaptionvpos{figure}{"c" oder "t" oder "b"}
% \begin{SCfigure}[rel. Breite (normalerweise = 1)][hbt]
%   \centering
%   \includegraphics[width=0.5\textwidth]{grafik.png}
%   \caption{Beschreibung}
%   \label{fig:}
% \end{SCfigure}
\usepackage{sidecap}

% Befehl für "Entspricht"-Zeichen
\newcommand{\corresponds}{\ensuremath{\mathrel{\widehat{=}}}}

%Fußnoten zwingend auf diese Seite setzen
\interfootnotelinepenalty=1000

%Für chemische Formeln (von www.dante.de)
%% Anpassung an LaTeX(2e) von Bernd Raichle
\makeatletter
\DeclareRobustCommand{\chemical}[1]{%
  {\(\m@th
   \edef\resetfontdimens{\noexpand\)%
       \fontdimen16\textfont2=\the\fontdimen16\textfont2
       \fontdimen17\textfont2=\the\fontdimen17\textfont2\relax}%
   \fontdimen16\textfont2=2.7pt \fontdimen17\textfont2=2.7pt
   \mathrm{#1}%
   \resetfontdimens}}
\makeatother


\begin{document}

\begin{titlepage}
\centering
\textsc{\Large Anfängerpraktikum der Fakultät für
  Physik,\\[1.5ex] Universität Göttingen}

\vspace*{4.2cm}

\rule{\textwidth}{1pt}\\[0.5cm]
{\huge \bfseries
  Versuch Diffusion\\[1.5ex]
  Protokoll}\\[0.5cm]
\rule{\textwidth}{1pt}

\vspace*{3cm}

\begin{Large}
\begin{tabular}{ll}
Praktikant: &  Michael Lohmann\\
% &  Felix Kurtz\\
Mitpraktikant: &  Kevin Lüdemann\\
% &  Skrollan Detzler\\
 E-Mail: & m.lohmann@stud.uni-goettingen.de\\
% &  felix.kurtz@stud.uni-goettingen.de\\
Mitpraktikant:  &  kevin.luedemann@stud.uni-goettingen.de\\
% &  skrollan.detzler@stud.uni-goettingen.de\\
 Betreuer: & Martin Ochmann\\
 Versuchsdatum: & 30.06.2014\\
\end{tabular}
\end{Large}

\vspace*{0.8cm}

\begin{Large}
\fbox{
  \begin{minipage}[t][2.5cm][t]{6cm} 
    Testat:
  \end{minipage}
}
\end{Large}

\end{titlepage}

\tableofcontents

\newpage

\section{Einleitung}
\label{sec:einleitung}
Die Diffusion (vom lateinischen diffundere 'sich ausbreiten') ist die Eigenschaft von verschiedenen Medien, sich ohne äußere Einflüsse miteinander zu vermischen bis ein Gleichgewichtszustand erreicht ist, nur dadurch, dass sie miteinander in Kontakt stehen.
So verursacht jede räumliche Inhomogenität einer physikalischen Größe einen Ausgleichsstrom.
Bei Materie geschieht dies aufgrund der statistischen Bewegung der Teilchen, der Braunschen Molekularbewegung.
Die Diffusion ist eine sehr wichtige Eigenschaft von Teilchen, da ohne sie zum Beispiel kein Leben existieren könnte, da kein Sauerstoff in unsere Zellen transportiert werden könnte.



\section{Theorie}
\label{sec:theorie}


\section{Durchführung}
\label{sec:durchfuehrung}
\subsection{Aufbau}
\begin{figure}
 \centering
 \def\svgwidth{0.8\columnwidth}
 \input{aufbau.pdf_tex}
 \caption{Versuchsaufbau\label{fig:aufbau}}
\end{figure}
Der Aufbau ist in Abb. \ref{fig:aufbau} dargestellt.
Das wichtigste Element hierbei ist eine Glasküvette, welche von einer Quecksilberdampflampe beleuchtet wird.
Das Licht der Lampe wird vorher von einer Blende und einer Linse gebündelt auf die Küvette mitsammt der darin enthaltenen Flüssigkeit gelenkt.
Nachdem es sie durchquert hat, nimmt eine Fotodiode das verbleibende Licht auf.
Der sich dadurch verändernde Wiederstand der Diode wird nun mithilfe der Wheatstonschen Brückenschaltung vermessen.\\

\subsection{Messung 1}
\label{sec:messung1}
Für Messung 1 wird der verstellbare Wiederstand solange angepasst, bis das Voltmeter keine Spannung mehr anzeigt, wenn der Graufilter $c_0/16$ eingelegt ist.
Dann wird die Messküvette zu 3/4 mit Wasser gefüllt und darüber Methylenblau der Konzentration $c_0$ geschichtet.
Nun wird die Küvette vorsichtig in das Stativ gesteckt und mit der Mikrometerschraube der Ort gesucht, an dem das Voltmeter ebenfalls keine Spannung zeigt.
Dieser Ort wird nun notiert und eine Stoppuhr gestartet.
Im Abstand von 30s wird nun immer wieder die jeweilige Höhe bestimmt, zu der die Konzentration der Flüssigkeit gerade $c_0/16$ beträgt.
Diese Werte werden notiert, bis die Werte von 30min aufgenommen sind.
Nach Beendigung der Messung wird die Stoppuhr weiterlaufen gelassen, sowie die Küvette vorsichtig entfernt und in einem Ständer für die spätere Weiterverwendung gelagert.

\subsection{Messung 2}
Messung 2 erfolgt analog zu Messung 1, nur mit einer Konzentration von $c_0/32$.
Dabei wird eine neue Küvette, sowie eine neue Stoppuhr benutzt.
Auch hier wird die Stoppuhr nach Beendigung der Messung weiter laufen gelassen.

\subsection{Messung 3}
Nach Beendigung von Messung 2 (ungefähr 40min nach Beginn) wird von der zweiten Küvette das Konzentrationsprofil ermittelt.
Dafür steckt man in den zweiten Ständer des Stativs nacheinander Graufilter verschiedener Intensität ($c_0/2, c_0/4, c_0/8, c_0/16, c_0/32$) und eicht jeweils die Wheatstonsche Brücke auf deren Helligkeit.
Dann fährt man das Stativ vorsichtig zur Küvette und bestimmt die entsprechende Höhe, in der die Konzentration gleich ist.
Dazu notiert man ebenfalls die Zeit nach Beginn von Messung 2, zu der dies abgelesen wird.
Dies tut man jeweils für auf- und absteigende Konzentrationen, bis man 10 Messwerte aufgenommen hat.

\subsection{Messung 4}
Messung 4 ist analog zu Messung 3, nur mit Küvette 1.
Sie erfolgt ca. 100min nach Beginn von Messung 1.


\section{Auswertung}
\label{sec:auswertung}

\section{Diskussion}
\label{sec:diskussion}
Trägt man die Messwerte von Messung 1 in einem Diagramm auf, so ist offensichtnach zwischen 14min und 14.5min ein Sprung in den Werten.
Dieser ist durch einen Stoß an den Tisch zu erklären, welcher die Vermischung der Flüssigkeiten in dem Moment extrem beschleunigt hat.
Um dies nicht mit zu berücksichtigen, teilten wir für die Geradensteigung die Messung in zwei Datensätze auf und nahmen den gewichteten Mittelwert der beiden.
\end{document}
